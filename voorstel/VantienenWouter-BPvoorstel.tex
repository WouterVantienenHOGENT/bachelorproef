%==============================================================================
% Sjabloon onderzoeksvoorstel bachproef
%==============================================================================
% Gebaseerd op document class `hogent-article'
% zie <https://github.com/HoGentTIN/latex-hogent-article>

% Voor een voorstel in het Engels: voeg de documentclass-optie [english] toe.
% Let op: kan enkel na toestemming van de bachelorproefcoördinator!
\documentclass{hogent-article}

% Invoegen bibliografiebestand
\addbibresource{voorstel.bib}

% Informatie over de opleiding, het vak en soort opdracht
\studyprogramme{Professionele bachelor toegepaste informatica}
\course{Bachelorproef}
\assignmenttype{Onderzoeksvoorstel}
% Voor een voorstel in het Engels, haal de volgende 3 regels uit commentaar
% \studyprogramme{Bachelor of applied information technology}
% \course{Bachelor thesis}
% \assignmenttype{Research proposal}

\academicyear{2025-2026} % TODO: pas het academiejaar aan

% TODO: Werktitel
\title{Real-time detectie van personendichtheid op de Gentse Feesten via beeldanalyse}

% TODO: Studentnaam en emailadres invullen
\author{Wouter Vantienen}
\email{wouter.vantienen@student.hogent.be}

\projectrepo{https://github.com/WouterVantienenHOGENT/bachelorproef}

% TODO: Medestudent
% Gaat het om een bachelorproef in samenwerking met een student in een andere
% opleiding? Geef dan de naam en emailadres hier
% \author{Yasmine Alaoui (naam opleiding)}
% \email{yasmine.alaoui@student.hogent.be}

% TODO: Geef de co-promotor op
% \supervisor[Co-promotor]{S. Beekman (Synalco, \href{mailto:sigrid.beekman@synalco.be}{sigrid.beekman@synalco.be})}

% Binnen welke specialisatierichting uit 3TI situeert dit onderzoek zich?
% Kies uit deze lijst:
%
% - Mobile \& Enterprise development
% - AI \& Data Engineering
% - Functional \& Business Analysis
% - System \& Network Administrator
% - Mainframe Expert
% - Als het onderzoek niet past binnen een van deze domeinen specifieer je deze
%   zelf
%
\specialisation{AI \& Data Engineering}
\keywords{Computervisie, Deep Learning, Druktecontrole}

\begin{document}

\begin{abstract}
   Grootschalige openluchtevenementen zoals de Gentse Feesten brengen aanzienlijke drukte met zich mee, wat voor veel uitdagingen zorgt voor hulpdiensten. De huidige methoden voor crowd control schieten vaak te kort, omdat ze subjectief, reactief of arbeidsintensief zijn. uit onderzoek blijkt dat er nood is aan een geautomatiseerd systeem dat real-time druktecontrole kan bieden. Daarom richt dit onderzoek zich op hoe computervisie ingezet kan worden om real-time, privacy-vriendelijk personendichtheid te voorspellen en te visualiseren. Tijdens het onderzoek gaan we een vergelijkende studie van bestaande modellen voeren. Daarna wordt er relevante data verzameld, waarop we de modellen trainen en evalueren. Op basis van de resultaten wordt een Proof of Concept ontwikkeld, gericht op de Gentse Feesten. Uiteindelijk is het verwachte resultaat een werkend systeem dat een verbeterde veiligheid kan bieden tijdens de Gentse Feesten door snelle identificatie van potentiële risicozones. Dit systeem zal gemakkelijk te integreren zijn met bestaande infrastructuur, heeft geen speciale hardware nodig, en voldoet aan de privacywetgeving.
\end{abstract}

\tableofcontents

% De hoofdtekst van het voorstel zit in een apart bestand, zodat het makkelijk
% kan opgenomen worden in de bijlagen van de bachelorproef zelf.
%---------- Inleiding ---------------------------------------------------------

% TODO: Is dit voorstel gebaseerd op een paper van Research Methods die je
% vorig jaar hebt ingediend? Heb je daarbij eventueel samengewerkt met een
% andere student?
% Zo ja, haal dan de tekst hieronder uit commentaar en pas aan.

%\paragraph{Opmerking}

% Dit voorstel is gebaseerd op het onderzoeksvoorstel dat werd geschreven in het
% kader van het vak Research Methods dat ik (vorig/dit) academiejaar heb
% uitgewerkt (met medesturent VOORNAAM NAAM als mede-auteur).
% 

\section{Inleiding}%
\label{sec:inleiding}

Grootschalige openluchtevenementen zoals de Gentse Feesten of festivals trekken jaarlijks honderdduizenden bezoekers naar de binnenstad. Deze soort evenementen brengen een aanzienlijke drukte met zich mee, wat voor uitdagingen zorgt op het vlak van openbare veiligheid. Incidenten uit het verleden uit andere landen leert dat het incorrect inschatten van drukte kan leiden tot levensgevaarlijke situaties zoals verdrukking en paniek.

De huidige methoden voor druktecontrole bij openluchtevenementen (zonder vaste toegangscontrolepunten) zijn vaak gebaseerd op handmatige tellingen en kennis van voorgaande evenementen. Deze methoden zijn vaak subjectief, reactief en hebben een grote vertraging tot het nemen van maatregelen. Een geautomatiseerd systeem kan in real-time een accurate inschatting van drukte bieden, waardoor organisatoren en hulpdiensten sneller en effectiever kunnen reageren op potentiële risico's.

Dit onderzoek richt zich op de organisatoren van de Gentse Feesten, en op de hulpdiensten die instaan voor de veiligheid tijdens deze feesten, zoals de lokale politie en brandweer.

De centrale onderzoeksvraag luidt: \emph{Hoe kan computervisie ingezet worden om real-time druktecontrole te realiseren tijdens openluchtevenementen zoals de Gentse Feesten?}

Het doel van deze bachelorproef is het ontwikkelen van een Proof of Concept (PoC). Dit systeem zal video-invoer van strategisch geplaatste camera's analyseren, en omzetten naar een Density Map. Het eindresultaat zal een balans zijn tussen nauwkeurigheid, verwerkingssnelheid, en nodige rekenkracht. Zo kan het systeem proactief hulpdiensten waarschuwen bij dreigende overbevolking.

%---------- Stand van zaken ---------------------------------------------------

\section{Literatuurstudie}%
\label{sec:literatuurstudie}

Een verkeerde inschatting van drukte kan leiden tot gevaarlijke situaties. Onderzoek door \textcite{Helbing2000} toont aan dat in drukke omgevingen, zoals tijdens grote evenementen, mensen onvoorspelbaar gedrag kunnen vertonen. Het onderzoek benadrukt het belang van een effectief model om drukte te voorspellen, en gevaarlijke situaties te voorkomen.

Uit onderzoek van \textcite{Zeitz2009} blijkt dat het begrijpen van gedrag van mensen in drukke omgevingen een serieuse uitdaging is. Er zijn verschillende factoren die het gedrag beïnvloeden. Het systeem moet in staat zijn om deze factoren te herkennen en te analyseren om zo nauwkeurige voorspellingen te kunnen maken.

De laatste jaren is er veel onderzoek gedaan naar computervisie en deep learning technieken. Maar volgens \textcite{Khan2020} blijft het een uitdaging om in real-time in ongewone situaties nauwkeurig drukte te voorspellen. Er is weinig data beschikbaar, en veel van de bestaande modellen zijn niet geoptimaliseerd voor drukte te voorspellen.

Uit een recent survey van \textcite{Gao2025} blijkt dat density estimation modellen een veelbelovende aanpak zijn voor druktecontrole en voorspellingen. In plaats van het tellen van individuele personen, zoals Convulutional Neural Networks (CNN's) doen, analyseren deze modellen de dichtheid van mensen in een bepaald gebied. Hierdoor zijn de modellen in staat om een density map te maken. Deze aanpak is efficiënter en kan beter omgaan met variaties in de omgeving. 

In dit onderzoek zullen we verder werken op de bevindingen van \textcite{Gao2025}. Het doel is om een PoC te ontwikkelen, en een van de bestaande density estimation modellen toe te passen. Verder houden we rekening met de nood aan real-time verwerking, zoals beschreven door \textcite{Khan2020}.

%---------- Methodologie ------------------------------------------------------
\section{Methodologie}%
\label{sec:methodologie}

Hier beschrijf je hoe je van plan bent het onderzoek te voeren. Welke onderzoekstechniek ga je toepassen om elk van je onderzoeksvragen te beantwoorden? Gebruik je hiervoor literatuurstudie, interviews met belanghebbenden (bv.~voor requirements-analyse), experimenten, simulaties, vergelijkende studie, risico-analyse, PoC, \ldots?

Valt je onderwerp onder één van de typische soorten bachelorproeven die besproken zijn in de lessen Research Methods (bv.\ vergelijkende studie of risico-analyse)? Zorg er dan ook voor dat we duidelijk de verschillende stappen terug vinden die we verwachten in dit soort onderzoek!

Vermijd onderzoekstechnieken die geen objectieve, meetbare resultaten kunnen opleveren. Enquêtes, bijvoorbeeld, zijn voor een bachelorproef informatica meestal \textbf{niet geschikt}. De antwoorden zijn eerder meningen dan feiten en in de praktijk blijkt het ook bijzonder moeilijk om voldoende respondenten te vinden. Studenten die een enquête willen voeren, hebben meestal ook geen goede definitie van de populatie, waardoor ook niet kan aangetoond worden dat eventuele resultaten representatief zijn.

Uit dit onderdeel moet duidelijk naar voor komen dat je bachelorproef ook technisch voldoen\-de diepgang zal bevatten. Het zou niet kloppen als een bachelorproef informatica ook door bv.\ een student marketing zou kunnen uitgevoerd worden.

Je beschrijft ook al welke tools (hardware, software, diensten, \ldots) je denkt hiervoor te gebruiken of te ontwikkelen.

Probeer ook een tijdschatting te maken. Hoe lang zal je met elke fase van je onderzoek bezig zijn en wat zijn de concrete \emph{deliverables} in elke fase?

%---------- Verwachte resultaten ----------------------------------------------
\section{Verwacht resultaat, conclusie}%
\label{sec:verwachte_resultaten}

Hier beschrijf je welke resultaten je verwacht. Als je metingen en simulaties uitvoert, kan je hier al mock-ups maken van de grafieken samen met de verwachte conclusies. Benoem zeker al je assen en de onderdelen van de grafiek die je gaat gebruiken. Dit zorgt ervoor dat je concreet weet welk soort data je moet verzamelen en hoe je die moet meten.

Wat heeft de doelgroep van je onderzoek aan het resultaat? Op welke manier zorgt jouw bachelorproef voor een meerwaarde?

Hier beschrijf je wat je verwacht uit je onderzoek, met de motivatie waarom. Het is \textbf{niet} erg indien uit je onderzoek andere resultaten en conclusies vloeien dan dat je hier beschrijft: het is dan juist interessant om te onderzoeken waarom jouw hypothesen niet overeenkomen met de resultaten.



\printbibliography[heading=bibintoc]

\end{document}