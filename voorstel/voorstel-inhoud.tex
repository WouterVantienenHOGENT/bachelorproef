%---------- Inleiding ---------------------------------------------------------

% TODO: Is dit voorstel gebaseerd op een paper van Research Methods die je
% vorig jaar hebt ingediend? Heb je daarbij eventueel samengewerkt met een
% andere student?
% Zo ja, haal dan de tekst hieronder uit commentaar en pas aan.

%\paragraph{Opmerking}

% Dit voorstel is gebaseerd op het onderzoeksvoorstel dat werd geschreven in het
% kader van het vak Research Methods dat ik (vorig/dit) academiejaar heb
% uitgewerkt (met medesturent VOORNAAM NAAM als mede-auteur).
% 

\section{Inleiding}%
\label{sec:inleiding}

\textcolor{red}{
  Grootschalige openluchtevenementen zoals de Gentse Feesten trekken jaarlijks hon\-derd\-dui\-zen\-den bezoekers naar de binnenstad. Dergelijke events zorgen voor enorme drukte, wat voor uitdagingen zorgt op het vlak van openbare veiligheid. Incidenten uit het verleden uit andere landen leren dat het incorrect inschatten van drukte kan leiden tot levensgevaarlijke situaties zoals verdrukking en paniek.
  De huidige methoden voor druktecontrole bij openluchtevenementen (zonder vaste toegangscontrolepunten) zijn vaak gebaseerd op handmatige tellingen en kennis van voorgaande evenementen. Deze methoden zijn vaak subjectief, reactief en hebben een grote vertraging tot het nemen van maatregelen. Een geautomatiseerd systeem kan in real-time een accurate inschatting van drukte bieden, waardoor organisatoren en hulpdiensten sneller en effectiever kunnen reageren op potentiële risico's.
}
Dit onderzoek richt zich op de organisatoren van de Gentse Feesten, en op de hulpdiensten die instaan voor de veiligheid tijdens deze feesten, zoals de lokale politie en brandweer.

De centrale onderzoeksvraag luidt: \textcolor{blue}{\emph{Hoe kan computervisie ingezet worden om real-time druk\-te\-con\-tro\-le te realiseren tijdens openluchtevenementen zoals de Gentse Feesten?}}

We kunnen deze centrale vraag opsplitsen in de volgende deelvragen:
\textcolor{teal}{
\begin{itemize}
    \item Welke factoren bepalen een gevaarlijke per\-so\-nen\-dicht\-heid, en hoe kunnen deze dienen als drempelwaarden voor waarschuwingen?
    \item Welke bestaande methoden voor druktecontrole zijn er, en wat zijn hun beperkingen?
    \item Welke computervisie- en deep learning-tech\-nie\-ken zijn het meest geschikt voor real-time druktecontrole?
    \item Welke datasets zijn beschikbaar voor het trainen en testen van computervisie-modellen gericht op druktecontrole?
\end{itemize}
}

Het doel van deze bachelorproef is het ontwikkelen van een Proof of Concept (PoC). Dit systeem zal video-invoer van strategisch geplaatste camera's analyseren, en omzetten naar een Density Map. Het eindresultaat zal een balans zijn tussen nauwkeurigheid, verwerkingssnelheid, en nodige rekenkracht. Zo kan het systeem proactief hulpdiensten waarschuwen bij dreigende overbevolking.

%---------- Stand van zaken ---------------------------------------------------

\section{Literatuurstudie}%
\label{sec:literatuurstudie}

Een verkeerde inschatting van drukte kan leiden tot gevaarlijke situaties. Onderzoek door \textcite{Helbing2000} toont aan dat in drukke omgevingen, zoals tijdens grote evenementen, mensen onvoorspelbaar gedrag kunnen vertonen. Het onderzoek benadrukt het belang van een effectief model om drukte te voorspellen, en gevaarlijke situaties te voorkomen.

Uit onderzoek van \textcite{Zeitz2009} blijkt dat het begrijpen van gedrag van mensen in drukke omgevingen een serieuze uitdaging is. Er zijn verschillende factoren die het gedrag beïnvloeden. Het systeem moet in staat zijn om deze factoren te herkennen en te analyseren om zo nauwkeurige voorspellingen te kunnen maken.

De laatste jaren is er veel onderzoek gedaan naar computervisie en deep learning technieken. Maar volgens \textcite{Khan2020} blijft het een uitdaging om in real-time in ongewone situaties nauwkeurig drukte te voorspellen. Er is weinig data beschikbaar, en veel van de bestaande modellen zijn niet geoptimaliseerd om drukte te voorspellen.

Uit een recent survey van \textcite{Gao2025} blijkt dat density estimation modellen een veelbelovende aanpak zijn voor druktecontrole en voorspellingen. In plaats van het tellen van individuele personen, zoals Convolutional Neural Networks (CNN's) doen, analyseren deze modellen de dichtheid van mensen in een bepaald gebied. Hierdoor zijn de modellen in staat om een density map te maken. Deze aanpak is efficiënter en kan beter omgaan met variaties in de omgeving. 

In dit onderzoek zullen we verder werken op de bevindingen van \textcite{Gao2025}. Het doel is om een PoC te ontwikkelen, en een van de bestaande density estimation modellen toe te passen. Verder houden we rekening met de nood aan real-time verwerking, zoals beschreven door \textcite{Khan2020}.

%---------- Methodologie ------------------------------------------------------
\section{Methodologie}%
\label{sec:methodologie}

Dit onderzoek zal in een eerste fase een grondige literatuurstudie zijn naar bestaande modellen voor druktecontrole op basis van computervisie en deep learning. We zullen de voor- en nadelen van de verschillende modellen analyseren, en vergelijken met elkaar. Dit wordt gevolgd door een technische uitwerking van een PoC. Hierbij zullen we een bestaand density estimation model verder optimaliseren voor gebruik tijdens de Gentse Feesten.

\textbf{Fase 1: Literatuur (maand 1): } We starten met een uitgebreide literatuurstudie om de huidige stand van zaken te begrijpen. Het recente survey van \textcite{Gao2025} zal als basis dienen. Verder gaan we ook dieper in op de kern van het probleem. We bestuderen welke factoren het gedrag van mensen in massa beïnvloeden, en hoe bestaande methoden hiermee omgaan. We analyseren waar de huidige methoden tekortschieten, en welke verbeteringen mogelijk zijn met computervisie. Op basis van deze inzichten kunnen we een shortlist maken van de meest veelbelovende modellen, die we in een volgende fase verder zullen uitwerken.

\textbf{Fase 2: Dataverzameling en voorbereiding (maand 2-3): } We gebruiken bestaande publieke datasets die relevant zijn voor druktecontrole, zoals de ShanghaiTech dataset. Daarnaast wordt er ook gekeken naar de mogelijkheid om aan data te komen van vorige edities van de Gentse Feesten, om zo tot een meer specifieke dataset te komen. Dit kan een meerwaarde zijn, maar het is niet noodzakelijk. Via data-aug\-men\-ta\-tie\-tech\-nie\-ken kunnen we de dataset verder uitbreiden, zodat het model robuuster wordt, en beter om kan gaan met variaties. Verder wordt er ook bekeken hoe we de camerabeelden kunnen anonimiseren, zodat we voldoen aan de GDPR-regelering.

\textbf{Fase 3: Ontwikkeling (maand 3-4): } Hier worden de gekozen modellen geïmplementeerd en getraind met de verzamelde data. Op basis van de literatuurstudie focussen we ons op density estimation modellen, zoals CSRNet en MCNN. We implementeren deze in Python met een framework zoals TensorFlow. Via hyperparameter tuning kunnen we de modellen verder optimaliseren, en de afweging maken tussen nauwkeurigheid en verwerkingssnelheid.

\textbf{Fase 4: Evaluatie (maand 4-5): } De prestaties van de verschillende modellen worden geëvalueerd op basis van nauwkeurigheid, verwerkingssnelheid en benodigde rekenkracht. We maken hierbij gebruik van metrieken zoals Mean Absolute Error (MAE) en Root Mean Squared Error (RMSE). Verder zijn scores zoals de F1-score en de Structural Similarity Index Measure (SSIM) belangrijk voor density maps. Uit deze evaluaties bepalen we welk model het meest geschikt is voor real-time druktecontrole tijdens de Gentse Feesten.

\textbf{Fase 5: Uitwerking (maand 5): } Op basis van de evaluatie kiezen we het beste model, en werken dit verder uit tot een volledig PoC. We documenteren het proces, en bereiden een presentatie om onze bevindingen en advies te delen met de betrokken doelgroep.

Gezien het iteratieve karakter van modelontwikkeling, is het mogelijk dat fases overlappen met elkaar. De planning is flexibel, zodat we kunnen inspelen op nieuwe inzichten die tijdens een fase naar boven komen.

%---------- Verwachte resultaten ----------------------------------------------
\section{Verwacht resultaat, conclusie}%
\label{sec:verwachte_resultaten}

Het verwachte resultaat is een PoC dat:
\begin{itemize}
  \item Real-time verwerking (> 15 FPS) haalt op standaard hardware.
  \item Een density map genereert met een nauwkeurigheid van MAE < 10 personen.
  \item Tegelijk 3-4 camerabeelden kan verwerken.
  \item Automatisch waarschuwingen kan generen bij dichtheden > 4 personen/m².
  \item Privacy-vriendelijk is door anonimisatie, en on-device verwerking zonder beeldopslag.
\end{itemize}

Het systeem zal gemakkelijk zijn om te integreren met bestaande infrastructuur, en zal geen speciale hardware vereisen. Verder zal het systeem voldoen aan de wetten rond privacy.

De meerwaarde van dit onderzoek ligt in het verbeteren van de veiligheid tijdens de Gentse Feesten. Door real-time inzicht te bieden in drukte en potentiële risicozones te identificeren, kunnen organisatoren en hulpdiensten sneller en efficiënter reageren. Dit kan helpen om gevaarlijke situaties te voorkomen, en de veiligheid van alle aanwezigen verbeteren.
